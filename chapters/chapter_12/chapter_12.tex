\documentclass[../../../main.tex]{subfiles}

\begin{document}
\setcounter{chapter}{11}
\chapter{Electrostatics}

\section{Coulomb's Law}

\begin{gather*}
    F=(\frac{1}{4\pi\epsilon_0})\frac{Q_1Q_2}{r^2}, \\
    \text{where } \epsilon_0= \text{permittivity of vacuum or free space}
\end{gather*}

\begin{mdframed}
    Definition: The magnitude of the electric force between two point charges is \emph{directly proportional to the product of the two charges} and \emph{inversely proportional to the square of the distance between them}.
\end{mdframed}
Given that F is inversely proportional to \(r^2\),
\begin{equation}
    F \propto \frac{1}{r^2}
\end{equation}
thus,
\begin{equation}
    \frac{F_1}{F_2}=(\frac{r_2}{r_1})^2
\end{equation}

\section{Electric Field}

\subsection{Electric Field}
\begin{mdframed}
    Definition: An electric field is a region in which an electric charge experiences a force. An electric field is represented by \textbf{electric field lines}.
\end{mdframed}

\pagebreak

\subsubsection{Electric Field Strength}
\begin{mdframed}
    Definition: The electric field strength at a point is the force per unit positive charge.
\end{mdframed}

The symbol for the electric field strength is \(E\) and it's unit is \(NC^{-1}\).
The direction of \(E\) at any point is the same as the direction of the force acting on a \textbf{positive test charge} placed at that point.

\begin{figure}[h]
    \includegraphics[scale=0.3]{figures/7.png}
    \centering
\end{figure}

If a charge Q is placed at a point which has an electric field intensity of E, then the force F acting on the charge is given by

\begin{equation}
    F=EQ
\end{equation}

\begin{figure}[h]
    \includegraphics[scale=0.3]{figures/8.png}
    \centering
\end{figure}

\subsection{Electric Field by a Point Charge}

\subsubsection{Derivation of electric field strength formula at a point in an electric field produced by a point charge Q.}

\begin{figure}[h]
    \includegraphics[scale=0.3]{figures/1.png}
    \centering
\end{figure}

\begin{align*}
    \text{Coulomb's Law: } F & =(\frac{1}{4\pi\epsilon_0})\frac{Q_1Q_2}{r^2}         \\
    F                        & =EQ_2                                                 \\
    E                        & =\frac{F}{Q_2}                                        \\
    E                        & =\frac{Q_1}{4\pi\epsilon_0r^2} \text{ (Substitute F)}
\end{align*}

\newpage

\subsubsection{Field Patterns}

\begin{figure}[h]
    \centering
    \subfloat[\centering Field pattern of a positive charge]{\includegraphics[scale=0.2]{figures/2.png}}
    \qquad
    \subfloat[\centering Field pattern of a negative charge]{\includegraphics[scale=0.2]{figures/3.png}}
    \caption{Field patterns of charges}
\end{figure}

\begin{figure}[h]
    \centering
    \subfloat[\centering Field pattern of a dipole]{\includegraphics[scale=0.5]{figures/4.png}}
    \qquad
    \subfloat[\centering Field pattern of two positive point charges]{\includegraphics[scale=0.6]{figures/5.jpeg}}
    \caption{Field patterns of dipoles}
\end{figure}

\newpage

\subsection{Motion of a Charged Particle in a Uniform Electric Field}

\begin{figure}[h]
    \centering
    \includegraphics[scale=0.6]{figures/6.png}
\end{figure}

Horizontally, the charged particle's acceleration is zero, and moves at a constant velocity, \(u_x\).
\begin{gather*}
    \text{Horizontally, } \\
    a_x =0,               \\
    v_x =u_x,               \\
    \text{whereby \(u_x\) is the initial velocity.}
\end{gather*}

Vertically, it is acted on by an upward force \(EQ\), having a constant upward acceleration of $\frac{EQ}{m}$.
\begin{gather*}
    \text{Vertically,} \\
    F=EQ,             \\
    F=ma,             \\
    a=\frac{EQ}{m}
\end{gather*}

\newpage

\subsubsection{Commonly used formulae}
\begin{mdframed}
    Time spent inside the electric field:
    \begin{equation}
        t=\frac{L}{u_x}
    \end{equation}
\end{mdframed}

\begin{mdframed}
    Vertical displacement of the electron, \(h\):
    \begin{align}
        s                      & =ut+\frac{1}{2}at^2                           \\
        \text{Substituting } a & =\frac{EQ}{m} \text{ and } t=\frac{L}{u_x}    \\
        h                      & =0+\frac{1}{2}(\frac{EQ}{m})(\frac{L}{u_x})^2 \\
        h                      & =\frac{1}{2}(\frac{EQ}{m})(\frac{L}{u_x})^2
    \end{align}
\end{mdframed}

\begin{mdframed}
    Vertical velocity of the electron at B:
    \begin{align}
        v_y & =u_y+at                        \\
        v_y & =0+\frac{EQ}{m}(\frac{L}{u_x})
    \end{align}
\end{mdframed}

\begin{mdframed}
    Horizontal velocity of the electron at B:
    \begin{align}
        At \text{ B, } v_x & =u_x
    \end{align}
    \begin{center}
        (Since there is no horizontal acceleration, the horizontal velocity remains constant.)
    \end{center}
\end{mdframed}

\begin{mdframed}
    Actual velocity of the electron at B:
    \begin{align}
        v=\sqrt{v_x^2+v_y^2}
    \end{align}
\end{mdframed}

\begin{mdframed}
    Direction of the motion of the electron at B:
    \begin{align}
        \tan{\theta} & =\frac{v_y}{v_x}
    \end{align}
\end{mdframed}

\pagebreak

\begin{mdframed}
    Total linear deflection of the electron on the screen, \(GC=h+y\):
    \begin{align*}
        h  & = \frac{1}{2}(\frac{EQ}{m})(\frac{L}{u_x})^2,               \\
        y  & = D \tan{\theta}                                            \\
        GC & = \frac{1}{2}(\frac{EQ}{m})(\frac{L}{u_x})^2+D \tan{\theta}
    \end{align*}
    \begin{center}
        \(h=\) the \emph{parabolic} motion of the electron while in the electric field

        \(y=\) the \emph{linear} deflection of the electron after exiting the electric field
    \end{center}
\end{mdframed}

\newpage

\section{Gauss's Law}
\subsection{Gauss's Law}

\begin{gather*}
    \Phi=\frac{Q}{\epsilon} \\
    \text{where } \epsilon= \text{permittivity of the medium}
\end{gather*}

\begin{mdframed}
    Definition: The \textbf{net} electric flux, \(\Phi\), passing through a \textbf{closed surface} (Gaussian surface) is equal to the total net charge Q inside the closed surface divided by permittivity, \(\epsilon\), of the medium.
\end{mdframed}
If the charge is in vacuum, then Gauss's law is written as
\begin{gather*}
    \Phi=\frac{Q}{\epsilon_0} \\
    \text{where } \epsilon_0= \text{permittivity of vacuum}
\end{gather*}

\subsection{Application of Gauss Law}

\subsubsection{Using Gauss's Law to determine the electric field strength for an isolated point charge/charged spherical conductor:}
\begin{align*}
    \text{Gauss's Law: } \Phi & =\frac{Q}{\epsilon_0}        \\
    \intertext{Because electric field intensity, \(E\) at a point can also be defined as the amount of electric flux per unit area (a.k.a. the electric flux density),}
    E                         & =\frac{\Phi}{A}              \\
    \intertext{\centering Substituting \(\Phi=\frac{Q}{\epsilon_0}\),}
    E                         & =\frac{Q}{\epsilon_0A}
    \intertext{If the charge is uniformly distributed over the surface of a sphere with radius \(r\),}
    \text{Hence, } E          & =\frac{Q}{4\pi\epsilon_0r^2} \\
\end{align*}
\emph{This formula can also be applied for an isolated point charge.}

\bigskip

However, it must be noted that this formula is only applicable to situations in which the charge is \textbf{uniformly distributed at the surface of a sphere}.

\pagebreak

Inside the sphere, the electric field strength is zero, as the electric field lines cancel each other out.
\begin{equation*}
    \text{Hence}, E=0
\end{equation*}

Outside the sphere, at a distance of \(r\) from the center of the sphere, the electric field strength is given by
\begin{equation*}
    E=\frac{Q}{4\pi\epsilon_0r^2}\
\end{equation*}

\subsubsection{Using Gauss's Law to determine the electric field strength for a uniformly charged conducting plate:}

\begin{figure}[h]
    \centering
    \includegraphics[scale=0.3]{figures/9.png}
\end{figure}

\begin{align*}
    \text{Gauss's Law: } \Phi & =\frac{Q}{\epsilon_0}      \\
    \intertext{Because electric field intensity, \(E\) at a point can also be defined as the amount of electric flux per unit area (a.k.a. the electric flux density),}
    E                         & =\frac{\Phi}{A}            \\
    \intertext{\centering Substituting \(\Phi=\frac{Q}{\epsilon_0}\),}
    E                         & =\frac{Q}{\epsilon_0A}
    \intertext{Assuming the charged plate (top surface of the cylinder) has a surface charge density of \(\sigma\) cm\(^{-2}\),}
    \sigma                    & =\frac{Q}{A}               \\
    Q                         & =\sigma A                  \\
    \text{Therefore, } E      & =\frac{\sigma}{\epsilon_0}
\end{align*}

\pagebreak

\section{Electric Potential}

\subsection{Electric Potential}

\begin{equation}
    V=\frac{W}{Q}
\end{equation}
\begin{align*}
    \text{Where } V & = \text{electric potential} \\
    W               & = \text{work done}          \\
    Q               & = \text{charge}
\end{align*}

\begin{mdframed}
    Definition: The electric potential at a point is the work done in bringing a unit charge from infinity to that point.
\end{mdframed}

The symbol for electric potential is \(V\) and its unit is \(JC^{-1}\) or \(V\).

\bigskip

% Electric potential is a \emph{scalar quantity}. The direction of the electric potential is from a point of high (\(\uparrow\)) electric potential to a point of low (\(\downarrow\)) electric potential.

\subsubsection{Electric Potential Energy of a charge Q}

\begin{equation}
    U=QV
\end{equation}
\begin{align*}
    \text{Where } U & = \text{electric potential energy} \\
    Q               & = \text{charge}                    \\
    V               & = \text{electric potential}
\end{align*}

\begin{mdframed}
    Definition: The work done in bringing a charge Q from infinity to a point.
\end{mdframed}

The symbol for electric potential energy is \(U\) and its unit is \(J\).

\subsubsection{Comparison between Electric Potential and Electric Potential Energy}

\begin{table}[h]
    \centering
    \begin{tabularx}{\linewidth}{|X|X|}
        \hline
        \textbf{Electric Potential}                                         & \textbf{Electric Potential Energy}                               \\
        \hline
        Work done in bringing a \emph{unit charge} from infinity to a point & Work done in bringing a \emph{charge Q} from infinity to a point \\
        \hline
        Scalar quantity                                                     & Scalar quantity                                                  \\
        \hline
        Symbol: V                                                           & Symbol: U                                                        \\
        \hline
        Unit: JC\(^{-1}\) or V                                              & Unit: J                                                          \\
        \hline
        \[V=\frac{W}{Q}\]                                                   & \[U=QV\]                                                         \\
        \hline
    \end{tabularx}
\end{table}

\pagebreak

\subsection{Potential Change, Potential Difference, \(\Delta V\)}

\begin{figure}[h]
    \centering
    \includegraphics[scale=0.3]{figures/10.png}
\end{figure}

\begin{mdframed}
    For electric potential (\(V\)),

    \centering
    The change in electric potential from A to B, \(\Delta V =V_B-V_A\) \\
    The change in electric potential from B to A,  \(\Delta V =V_A-V_B\)
\end{mdframed}

\begin{mdframed}
    For electric potential energy (\(U\)),

    {\centering
            \(U_A=QV_A\) \\
            \(U_B=QV_B\) \\
        }

    \bigskip

    When a charge Q moves from A to B, the charge has a change of electric potential energy, \(\Delta U\). Therefore, the electric potential energy change is given by
    \begin{align*}
        \Delta U & =U_B-U_A    \\
                 & =QV_B-QV_A  \\
                 & =Q(V_B-V_A) \\
        \Delta U & =Q\Delta V
    \end{align*}
\end{mdframed}

\begin{mdframed}
    When charge Q is moved from A to B,
    \begin{equation}
        W=\Delta U =Q\Delta V =Q(V_B-V_A)
    \end{equation}

    When a charge is moved from B to \(\infty\),
    \begin{equation}
        W=\Delta U =Q\Delta V =Q(V_{\infty}-V_B)
    \end{equation}
    Since the electric potential at \(\infty\) is zero, the equation simplifies to
    \begin{equation}
        W=-QV_B
    \end{equation}
    For a charge moved from \(\infty\) to B,
    \begin{align*}
        W & =Q(V_B-V_{\infty}) \\
        W & =QV_B
    \end{align*}
\end{mdframed}

Comparison of work done when work is done by external force and when work is done by the electric field force:

\begin{center}
    \begin{tabularx}{\linewidth}{|X|X|}
        \hline
        \textbf{Work done by external force}       & \(W_{\text{external}}=Q\Delta V\)        \\
        \hline
        \textbf{Work done by electric field force} & \(W_{\text{electric field}}=-Q\Delta V\) \\
        \hline
    \end{tabularx}
\end{center}

\end{document}